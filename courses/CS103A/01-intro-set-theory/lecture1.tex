\documentclass[11pt]{article}
\usepackage[margin=1in]{geometry}
\usepackage{amsmath,amssymb,amsthm}
\usepackage{enumitem}
\usepackage[T1]{fontenc}
\usepackage[utf8]{inputenc}
\usepackage{lmodern}

\title{CS103/CS103A --- Lecture 1: Introduction to Set Theory \\ \& Computation Foundations}
\author{Vedant}
\date{}

\begin{document}
\maketitle

\section*{Learning Goals}
\begin{itemize}[leftmargin=*]
  \item Understand the scope of CS103: computability theory, complexity theory, discrete mathematics
  \item Learn fundamental set theory concepts and notation
  \item Understand infinite sets and different sizes of infinity
  \item Recognize the relationship between programs, strings, and problems
  \item Appreciate the limits of computation (some problems are unsolvable)
\end{itemize}

\section*{Key Formulas \& Notation}
\begin{itemize}[leftmargin=*]
  \item Set notation: $\{a,b,c\}$ --- unordered collection of distinct elements
  \item Element notation:  \in S$ (\emph{x} is an element of $),  \notin S$ (\emph{x} is not an element)
  \item Equality of sets: same elements, order irrelevant, duplicates ignored
  \item Special sets: Empty set $\varnothing$; Naturals $\mathbb{N} = \{0,1,2,\dots\}$; Integers $\mathbb{Z} = \{\dots,-1,0,1,\dots\}$; Reals $\mathbb{R}$
  \item Set-builder notation: $\{x \in S \mid \text{property}(x)\}$, e.g., $\{n \in \mathbb{N} \mid n \text{ is even}\} = \{0,2,4,\dots\}$
  \item Common operations: Union  \cup B$, Intersection  \cap B$, Difference  \setminus B$, Symmetric difference  \triangle B$
  \item Subsets:  \subseteq T$ if every element of $ is in $. Power set: $\mathcal{P}(S) = \{ T \mid T \subseteq S\}$
  \item Cardinality: $|S|$ is the number of elements in $. Infinite cardinality: $|\mathbb{N}| = \aleph_0$
  \item Cantor's Theorem: $|S| < |\mathcal{P}(S)|$ --- every set is smaller than its power set
\end{itemize}

\section*{Notes}
\subsection*{Course Focus}
\begin{itemize}[leftmargin=*]
  \item Explore `laws of physics'' in CS --- fundamental limits on what computers can and cannot do
  \item Emphasis on proof-based reasoning, not just calculations
  \item No advanced math prerequisites; high school algebra is enough
\end{itemize}

\subsection*{Sets}
\begin{itemize}[leftmargin=*]
  \item Sets are unordered, contain distinct elements, and can include other sets
  \item Empty set $\varnothing$ has no elements; $\{\varnothing\}$ is a set containing the empty set
  \item Important distinction: an element vs. a set containing that element, e.g.,  \neq \{1\}$, $\varnothing \neq \{\varnothing\}$
  \item Duplicates in set definitions are ignored
  \item Infinite sets like $\mathbb{N}$, $\mathbb{Z}$, and $\mathbb{R}$ exist; not all infinities are equal in size
\end{itemize}

\subsection*{Infinite Cardinalities}
\begin{itemize}[leftmargin=*]
  \item Same-size rule: two sets have the same size if their elements can be paired 1:1 with none left over
  \item $\mathbb{N}$ and the even numbers have the same size (bijection  \leftrightarrow 2n$)
  \item $\mathbb{N}$ and $\mathbb{Z}$ have the same size (pair non-negatives with positives, negatives with odds)
  \item Cantor's diagonalization: $\mathcal{P}(S)$ is strictly larger than $ $\Rightarrow$ not all infinities are equal; there are infinitely many infinities
\end{itemize}

\subsection*{Programs, Strings, and Problems}
\begin{itemize}[leftmargin=*]
  \item Every computer program is a finite string of characters; hence $|\text{Programs}| \le |\text{Strings}|$
  \item Problems can be represented as sets of strings (e.g., given a string, determine membership in $)
  \item From Cantor: $|\text{Strings}| < |\mathcal{P}(\text{Strings})| \le |\text{Problems}|$
  \item Therefore $|\text{Programs}| < |\text{Problems}|$ $\Rightarrow$ there are more problems than programs; some problems are unsolvable
  \item For a random problem, probability of solvability is essentially 0
\end{itemize}

\section*{Examples}
\subsection*{Even Natural Numbers}
\begin{align*}
E 
  &= \{ n \in \mathbb{N} \mid n \text{ is even} \} \\
  &= \{0,2,4,6,\dots\}
\end{align*}
Cardinality: $|E| = |\mathbb{N}| = \aleph_0$ (same size as naturals despite being `smaller'' in content).

\subsection*{Set Relationships}
\begin{itemize}[leftmargin=*]
  \item  = \{1,2,3\}$,  = \{3,4,5\}$
  \item Union:  \cup B = \{1,2,3,4,5\}$
  \item Intersection:  \cap B = \{3\}$
  \item Difference:  \setminus B = \{1,2\}$
  \item Symmetric difference:  \triangle B = \{1,2,4,5\}$
\end{itemize}

\subsection*{Subset \& Power Set Example}
Given  = \{a,b\}$,
\[
\mathcal{P}(S) = \{\varnothing, \{a\}, \{b\}, \{a,b\}\}.
\]

\subsection*{Infinite Set Mapping (Bijection)}
A bijection between $\mathbb{N}$ and the even numbers:
\[
0 \leftrightarrow 0,\; 1 \leftrightarrow 2,\; 2 \leftrightarrow 4,\; 3 \leftrightarrow 6,\; \dots
\]

\subsection*{Cantor's Diagonalization (Conceptual)}
\begin{enumerate}[leftmargin=*]
  \item List all subsets of a set (conceptually)
  \item Flip the  element's membership in the  subset
  \item The resulting subset is not in the list $\Rightarrow$ $|\mathcal{P}(S)| > |S|$
\end{enumerate}

\section*{Reflection}
\begin{itemize}[leftmargin=*]
  \item Sets are the foundation for understanding computation limits
  \item Cantor's theorem connects set theory directly to the impossibility of solving certain problems with computers
  \item Computation isn't just about algorithms --- it's also about the inherent boundaries of what's possible
\end{itemize}

\end{document}
